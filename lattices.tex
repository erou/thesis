In Chapter~\ref{chap:isomorphism}, we have seen algorithms to compute
isomorphisms, or inclusions, between pairs of finite fields. Now, we want to
integrate these algorithms in a larger global system with potentially as many
finite fields as we want.
\minitoc

\begin{figure}%[h]
  \centering

    \tikzset{
        dotstyle/.style={circle, inner sep = 1.2pt, outer sep = 4pt, fill =
        gray},
        edgetower/.style={thick},
        edgecomp/.style={thick, lightgray}
          }
          \begin{tikzpicture}[scale=.9]
    \coordinate (T2) at (-2, 0.5);
    \node (Fp) at (0, 0) {$\mathbb{F}_p$};
    \node (Fp2) at ($(Fp) + (T2)$) {$\mathbb{F}_{p^2}$};
    \node (Fp4) at ($(Fp2) + (T2)$) {$\mathbb{F}_{p^4}$};
    \node (Fp2l) at ($(Fp4) + (T2)$) {};% {$\FF_p^{(2)}$};
    % ---------------------
    \coordinate (T3) at (-0.7, 2);
    \node (Fp3) at ($(Fp) + (T3)$) {$\mathbb{F}_{p^3}$};
    \node (Fp9) at ($(Fp3) + (T3)$) {$\mathbb{F}_{p^9}$};
    \node (Fp3l) at ($(Fp9) + (T3)$) {};% {$\FF_p^{(3)}$};
    % ---------------------
    \coordinate (T5) at (0.7, 2);
    \node (Fp5) at ($(Fp) + (T5)$) {$\mathbb{F}_{p^5}$};
    \node (Fp25) at ($(Fp5) + (T5)$) {$\mathbb{F}_{p^{25}}$};
    \node (Fp5l) at ($(Fp25) + (T5)$) {};% {$\FF_p^{(5)}$};
    % ---------------------
    \coordinate (Tl) at (2, .5);
    \node (Fpl) at ($(Fp) + (Tl)$) {$\mathbb{F}_{p^\ell}$};
    \node (Fpl2) at ($(Fpl) + (Tl)$) {$\mathbb{F}_{p^{\ell^2}}$};
    \node (Fpll) at ($(Fpl2) + (Tl)$) {};% {$\FF_p^{(\ell)}$};
    % ---------------------
    \node[dotstyle] (dot1) at ($(Fp2) + (Fp3) - (Fp)$) {};
    \node[dotstyle] (dot2) at ($(Fp4) + (dot1) - (Fp2)$) {};
    \node[dotstyle] (dot3) at ($(Fp2) + (Fp5) - (Fp)$) {};
    \node[dotstyle] (dot4) at ($(Fp3) + (Fp5) - (Fp)$) {};
    \node[dotstyle] (dot5) at ($(Fp3) + (Fpl) - (Fp)$) {};
    \node[dotstyle] (dot6) at ($(Fp5) + (Fpl) - (Fp)$) {};
    \node[dotstyle] (dot7) at ($(Fpl2) + (dot6) - (Fpl)$) {};
    % ---------------------
    \draw
    (Fp)
    edge[edgetower] (Fp2)
    edge[edgetower] (Fp3)
    edge[edgetower] (Fp5)
    edge[edgetower] (Fpl)
    (Fp2)
    edge[edgetower] (Fp4)
    edge[edgecomp] (dot1)
    (Fp4)
    edge[edgetower, dotted] (Fp2l)
    edge[edgecomp] (dot2)
    (dot1)
    edge[edgecomp] (dot2)
    (Fp3)
    edge[edgetower] (Fp9)
    edge[edgecomp] (dot1)
    edge[edgecomp] (dot4)
    (Fp9)
    edge[edgetower, dotted] (Fp3l)
    (Fp5)
    edge[edgetower] (Fp25)
    edge[edgecomp] (dot4)
    edge[edgecomp] (dot6)
    (Fp25)
    edge[edgetower, dotted] (Fp5l)
    (Fpl)
    edge[edgetower] (Fpl2)
    edge[edgecomp] (dot6)
    (Fpl2)
    edge[edgetower, dotted] (Fpll)
    edge[edgecomp] (dot7)
    (dot3)
    edge[edgecomp] (Fp2)
    edge[edgecomp] (Fp5)
    (dot5)
    edge[edgecomp] (Fp3)
    edge[edgecomp] (Fpl)
    (dot6)
    edge[edgecomp] (dot7);
  \end{tikzpicture}
  \caption{Extensions de $\mathbb{F}_p$.}
  \label{fig:alg-closure}
\end{figure}

\clearpage
\section{The compatibility problem}

Now that we know how to go from one finite field $\mathbb{F}_{p^a}$ to another
$\mathbb{F}_{p^b}$, with $p\in\mathbb{N}$ a prime number and
\[
  a\,|\,b
\]
two integers, we would like to be able to manage more than $2$ finite fields
simultaneously. In other words, given a family
\[
  \F=\left\{ \mathbb{F}_{p^{a}}\,|\,a\in E \right\}
\]
for $E$ a subset of $\mathbb{N}\setminus\left\{ 0 \right\}$,
we want to be able to compute an embedding
\[
  \mathbb{F}_{p^a}\emb\mathbb{F}_{p^b}
\]
each time that we have $\mathbb{F}_{p^a}, \mathbb{F}_{p^b}\in\F$ with $a$
dividing $b$. On a pratical point of view, we want to build a computer algebra
system where the users can embed the field they are working with in a bigger
finite field, or conversely if an element is known to belong to a smaller field
than the ambient field, project it to a smaller field. In cryptology and coding
theory, finite fields are ubiquitus, and some algorithms require frequent change
of field. For example, in the
quasi-polynomial algorithm for discrete logarithm in small characteristic by
Granger, Kleinjung and Zumbrägel~\cite{GKZ14}, we have to work with a tower of
finite field extensions, and thus a computer algebra system automatically
dealing with the changes would be very convenient in order to implement their
algorithm. On a theoretical point of view, this leads to the question of the
arithmetic in the algebraic closure of some finite field $\mathbb{F}_p$
\[
  \bar{\mathbb{F}}_p=\bigcup_{j\in\mathbb{N}\setminus\left\{ 0
  \right\}}\mathbb{F}_{p^j}
\]
and its representation on a computer, a question that was for example
investigated in~\cite{DDS14}. 

If we \emph{just} want to be able to compute embeddings between two finite
fields in $\F$ when it makes sense, we can just use one of the algorithms presented
in Chapter~\ref{chap:isomorphism} each time we need it. In fact, we want to
build a data structure $\Lambda$ to represent all the extensions of
$\mathbb{F}_p$ that we need, and additionnal sub-goals might be desirable.
\begin{description}
\item[\emph{Effective embeddings:}] for any pair of extensions
  $k\subset K$ in $\Lambda$, there exists an efficiently computable
  embedding $\phi:k\to K$, and algorithms to evaluate $\phi$ on $k$,
  and the section $\phi^{-1}$ on $K$.
\item[\emph{Compatibility:}] the embeddings are \emph{compatible},
  \ie for any triple $k\subset K\subset L$ in $\Lambda$, and
  embeddings $\phi:k\to K$, $\psi:K\to L$, $\chi:k\to L$ such as shown in
  Figure~\ref{fig:compatibility}, one has
  $\chi=\psi\circ\phi$.
  \begin{figure}[h]
    \centering
    \begin{tikzpicture}
      \node (E) at (0, 0) {$k$};
      \node (F) at (1.5, 1) {$K$};
      \node (G) at (0.5, 2) {$L$};

      \draw[arrow] (E) -- (F);
      \draw[arrow] (E) -- (G);
      \draw[arrow] (F) -- (G);

      \node (f12) at (1, 0.25)
      {$\phi$};
      \node (f13) at (-0.1, 1)
      {$\chi$};
      \node (f23) at (1.4, 1.65)
      {$\psi$};
    \end{tikzpicture}

  \caption{Embeddings between finite fields $k\subset K\subset L$.}
  \label{fig:compatibility}
  \end{figure}
\item[\emph{Incrementality:}] the data associated with an extension
  (\eg its irreducible polynomial, change-of-basis matrices, \dots)
  must be computable efficiently and \emph{incrementally}, \ie 
  adding a new field extension to $\Lambda$ does not require
  recomputing data for all extensions already in $\Lambda$.
\item[\emph{Uniqueness:}] any extension of $\mathbb{F}_{p}$ is determined by an
  irreducible polynomial whose definition only depends on the
  characteristic $p$ and the degree of the extension.
\item[\emph{Generality:}] extensions of $\mathbb{F}_{p}$ can be represented by
  arbitrary irreducible polynomials.
\end{description}
We cannot fulfill all these conditions simultaneously, as
uniqueness and generality are in conflict with each other. One might prefer to
look for uniqueness or generality, depending on the situation, but it would
always be better to have effective embeddings, compatibility and incrementality.
In order to achieve the ``effective embeddings'' condition, we can use the
efficient algorithms presented in Chapter~\ref{chap:isomorphism}. The only
problem is that if we have three finite fields
\[
  k\subset K\subset L
\]
and three embeddings $\phi:k\to K$, $\psi:K\to L$, $\chi:k\to L$, there is no
warranty that the diagram of Figure~\ref{fig:compatibility} commutes, \ie
\[
  \chi = \psi\circ\phi.
\]
If we want to achive compatibility, we have to provide extra-work.
The two constructions presented in Sections~\ref{sec:conway}
and~\ref{sec:bosma-canon-steel} are both solutions oriented towards this goal.
Conway polynomials allow \emph{uniqueness} while Bosma, Cannon and Steel
framework allows \emph{generality}.

\section{Conway polynomials}
\label{sec:conway}

% TODO
% ====
%
% Give a reference for the fact that Conway polynomials always exist?

Conway polynomials~\cite{Parker90, Scheerhorn92} are named after J. H. Conway,
who suggested the idea of working with norm-compatible polynomials when
constructing extension fields. They were introduced by Parker in 1990 and
provide embeddings that are easy to compute.

\begin{defi}[Conway polynomials]
  Let $p\in\mathbb{N}$ be a prime number. The $m$-th \emph{Conway polynomial}
  \[
    C_m\in\mathbb{F}_p[X]
  \]
  is the lexicographically smallest monic irreducible polynomial of degree $m$
  that is also primitive (\ie its roots generate $\mathbb{F}_{p^m}^\times$) and
  norm-compatible, \ie if
  \[
    m\,|\,n
  \]
  we have
  \[
    C_m(X^{\frac{p^n-1}{p^m-1}})\mod C_n = 0.
  \]
\end{defi}
The norm compatibility means that if one takes the norm relatively to the
extension 
\[
  \mathbb{F}_{p^n}/\mathbb{F}_{p^m}
\]
of a root of $C_n$, one obtains a root of $C_m$. This compatibility will allow
us to define compatible embeddings. Let $p\in\mathbb{N}$ be a prime number, $m,
n\in\mathbb{N}$ be two integers such that
\[
  m\,|\,n
\]
and let $C_m, C_n\in\mathbb{F}_{p}[X]$ be the $m$-th and the $n$-th Conway
polynomial. Let 
\[
  k=\mathbb{F}_{p^m}=\mathbb{F}_p(\alpha_m)\cong\mathbb{F}_{p}[X]/(C_m)
\]
and
\[
  K = \mathbb{F}_{p^n}=\mathbb{F}_p(\alpha_n)\cong\mathbb{F}_{p}[X]/(C_n)
\]
with $\alpha_m=\bar X\in k$ and $\alpha_n=\bar X\in K$. Now, we define the
embedding
\[
\begin{array}{cccl}
  \embed{k}{K}:&k&\emb &K\\
  & \alpha_m&\mapsto&(\alpha_n)^{\frac{p^n-1}{p^m-1}}
\end{array}
\]
that sends $\alpha_m$ to $(\alpha_n)^{\frac{p^n-1}{p^m-1}}$, the norm of
$\alpha_n$ relative to $k$. Because this is a field morphism, it is implied that
every polynomial in $\alpha_m$ is sent to the same polynomial in
$(\alpha_m)^{\frac{p^n-1}{p^m-1}}$. That is indeed an embedding because of the
norm-compatibility condition in Conway polynomials. This definition also
provides compatibility thanks to the transitivity of the norm: let
\[
  k\subset K\subset L
\]
three finite fields defined using Conway polynomials, let
\[
  N_{K/k}, N_{L/K}, N_{L/k}
\]
be the respective norms of the extensions
\[
  K/k, L/K, L/k
\]
and
\[
  \embed{k}{K}, \embed{K}{L}, \embed{k}{L}
\]
the corresponding embeddings. Then, for any element $x\in L$, we have
\[
  N_{L/k}(x) = N_{K/k}(N_{L/K}(x)),
\]
and it follows, using the fact that $\embed{K}{L}$ and $N_{K/k}$ commute, that
\[
  \embed{k}{L}=\embed{K}{L}\circ\embed{k}{K}.
\]
Therefore, using Conway polynomials to define all finite field extensions allows
compatibility \emph{a priori}: adding a new extension in the lattice does not
require any computation and the form of the embedding is already known. The
only condition is to define each extension using the Conway polynomial of the
corresponding degree. These polynomial always exist so this is possible,
although the best known algorithm to compute Conway polynomials has exponential
complexity~\cite{HL98}. Because the cost of computing Conway polynomials is
prohibitive, they are often precomputed and tabulated up to a certain degree in
many computer algebra systems. As a result, most computer algebra systems use
Conway polynomial to represent finite fields of small degrees, and switch to
other representations when Conway polynomials are no longer available. This
strategy leads to efficient lattices of compatibly embedded finite fields, at
least in small degrees, but at the cost of generality, because the finite fields
extensions cannot be user-defined and high degree extensions cannot be included
in the lattice. We see in Section~\ref{sec:bosma-canon-steel} that those issues
can be adressed, but the price to compute the embeddings is higher.

\section{Bosma-Canon-Steel framework}
\label{sec:bosma-canon-steel}

In 1987, Bosma, Canon and Steel proposed a new framework~\cite{BCS97} to deal
with lattices of finite field extensions. These algorithms were initially
implemented in the computer algebra system MAGMA. The framework allows the user
to define finite fields with custom polynomials, and compute the
correct embeddings on the fly, in a way that guarantees that at least one
compatible embedding will always exist between two finite fields, when that
makes sense. To the best of my knowledge, this framework was only used in MAGMA
until very recently. We first implemented the framework using Nemo and Flint and
presented our software in the symbolic computation conference ISSAC in
2018~\cite{DRR18}. In 2019, the code was also added to the official branch of
Nemo, and Nemo can now deal with finite field extensions and compatibly embed
them, in the case where the characteristic $p$ of the fields fit in a machine
word. There are no theretical obstacles preventing from implementing the algorithm in arbitrary
large characteristic, but some tools were lacking at the C level (\ie in
Flint).

\subsection{Bosma-Canon-Steel algorithm}
\paragraph{Notations and first results.} Let $p\in\mathbb{N}$ be a prime number
and $q$ a power of $p$. We present the results over the finite field
$\mathbb{F}_q$, even if the applications we have in mind apply to prime field
extensions. Given two finite fields $k$ and $K$ with cardinalities
\[
  \Card k = q^m
\]
and
\[
  \Card K =  q^n
\]
such that $m$ divides $n$, we know that $k$ can be embedded in $K$. In other
words, we know that $K$ contains one subfield
\[
  k'=\left\{ x\in K\,|\,x^{q^m}=x \right\}\subset K
\]
with 
\[
  \Card k' = q^m.
\]
There is exactly one subfield $k'$ in $K$ that has the same cardinality as $k$,
but there exist $m$ different ways to map $k$ to $k'$. One way of seeing that is
to say that an element $\alpha\in k$ generating $k$ has a minimal polynomial
$\pi$ over $\mathbb{F}_q$ of degree $m$, and that $\alpha$ can be mapped to any
one of the $m$ roots of $\pi$ in $K$. This choice sets the embedding because any
element in $k$ can be written as
\[
  \sum_{j=0}^{m-1}a_j \alpha^j,
\]
with $a_j\in\mathbb{F}_q$. We write $k\emb K$ if an explicit embedding has been
chosen. Let
\[
  k\subset K\subset L
\]
with $k\emb K$ and $k\emb L$, as in Figure~\ref{fig:uncomplete}. The fundations
of Bosma-Canon-Steel framework lay on the fact that we precisely know how many
compatible embeddings from $K$ to $L$ we can find.
  \begin{figure}%[h]
    \centering
    \begin{tikzpicture}
      \node (E) at (0, 0) {$k$};
      \node (F) at (1.5, 1) {$K$};
      \node (G) at (0.5, 2) {$L$};

      \draw[arrow] (E) -- (F);
      \draw[arrow] (E) -- (G);
      \draw[dashed-arrow] (F) -- (G);

      \node (f12) at (1, 0.25) {$\phi$};
      \node (f13) at (-0.1, 1) {$\chi$};
      %\node (f23) at (1.4, 1.65) {$\psi$};
    \end{tikzpicture}
  \caption{Uncomplete embedding diagram.}
  \label{fig:uncomplete}
  \end{figure}
  \begin{prop}[{\cite[Thm.~2.1]{BCS97}}]
    \label{prop:number-embeddings}
    Let $l\,|\,m\,|\,n$ three integers and
    \[
      k\subset K\subset L
    \]
    three finite fields extensions of $\mathbb{F}_q$, with respective
    cardinalities $q^l$, $q^m$ and $q^n$. Assume that $k\emb K$ and $k\emb L$,
    then there are $m/l$ compatible embeddings from $K$ to $L$.
  \end{prop}
  \begin{proof}
    Let $\phi:k\to K$ be the embedding from $k$ to $K$ and $\chi:k\to L$ the
    embedding from $k$ to $L$. Let $\widetilde\psi:K\to L$ be any embedding from $K$ to
    $L$. The maps $\chi$ and $\widetilde\psi\circ\phi$ are two embeddings from
    $k$ to $L$. Since there are only $l$ such maps, and since the difference
    between them is only a matter of which conjugate is chosen for a generating
    element in $k$, we know that there exist
    \[
      \sigma\in\Gal(L/\mathbb{F}_q)
    \]
    such that
    \[
      \chi = \sigma\circ\widetilde\psi\circ\phi.
    \]
    If we set
    \[
      \psi=\sigma\circ\widetilde\psi,
    \]
    we obtain a compatible embedding $\psi:K\to L$ from $K$ to $L$. If we
    compose $\psi$ with a $\chi(k)$-automorphism of $L$, \ie an automorphism of
    $\psi(K)$ that
    fixes the subfield $\chi(k)$, we still obtain a compatible embedding because
    the images of the elements in $\chi(k)$ are unchanged. Now, we know that
    \[
      \Card\Gal(\psi(K)/\chi(k)) = m/l,
    \]
    thus we know that there are exactly $m/l$ embeddings from $K$ to $L$ that
    are compatible with $\phi$ and $\chi$.
  \end{proof}
  In the future, if we have two finite fields $k$ and $K$ such that $k\emb K$
  with an explicit embedding $\phi:k\to K$, we will often identify $k$ with
  its isomorphic image $\phi(k)\cong k$ in $K$, for the sake of simplicity. We
  also have a constructive version of Proposition~\ref{prop:number-embeddings}.
\begin{prop}[{\cite[Thm.~2.2]{BCS97}}]
    \label{prop:number-embeddings}
    Let $l\,|\,m\,|\,n$ three integers and
    \[
      k\subset K\subset L
    \]
    three finite fields extensions of $\mathbb{F}_q$, with respective
    cardinalities $q^l$, $q^m$ and $q^n$. Assume that $k\emb K$ and $k\emb L$,
    and let $\phi=\embed{k}{K}$, $\chi=\embed{k}{L}$ the respective embeddings.
    Let also $\alpha\in K$ be an element that generates $K$ over $k$ and
    \[
      \pi_k=\Minpoly_k(\alpha)
    \]
    the minimum polynomial of $\alpha$ over $k$. Let $\rho\in L$ be a root of
    $\pi_k$ in $L$. Let $\psi:K\to L$ be the map defined by
    \[
      \psi(\sum_{j=0}^{m/l-1}x_j\alpha^j) = \sum_{j=0}^{m/l}x_j\rho^j,
    \]
    using the unique representation of any element of $K$ as $\sum_j x_j
    \alpha^j$ with $x_j\in k$. Then $\psi$ is an embedding from $K$ to $L$ that
    is compatible with $\phi$ and $\chi$, \ie we have
    \[
      \chi = \psi\circ\phi.
    \]
  \end{prop}
  \begin{proof}
    The root $\rho$ in $L$ exist since we know that $\pi_k$ has a root in $K$
    and $K\subset L$. The map $\psi$ is then well-defined because of the uniqueness of the
    representation of the elements in $K$ as $\sum_jx_j\alpha^j$. The elements
    $\alpha$ and $\rho$ have the same minimum polynomial so the map $\psi$ is a
    homomorphism. The map $\psi$ is also injective and thus it is an embedding
    from $K$ to $L$. The compatibility condition holds by construction.
  \end{proof}
  Although the result of Proposition~\ref{prop:number-embeddings} captures only
  one configuration among the three different possibilities of
  Figure~\ref{fig:triangles}, it is the most important case. Indeed, if we
  already know $\embed{k}{K}$ and $\embed{K}{L}$ (the configuration in the
  middle of Figure~\ref{fig:triangles}), we can just set
  \[
    \embed{k}{L} = \embed{K}{L}\circ\embed{k}{K}.
  \]
  Finally the configuration on the right, where we know $\embed{k}{L}$ and
  $\embed{K}{L}$, is also unrelevant because, by construction, this case never
  happens when working with Bosma-Canon-Steel framework.
    \begin{figure}
    \centering
    \begin{tikzpicture}
      \node (E) at (0, 0) {$k$};
      \node (F) at (1.5, 1) {$K$};
      \node (G) at (0.5, 2) {$L$};

      \draw[arrow] (E) -- (F);
      \draw[arrow] (E) -- (G);
      \draw[dashed-arrow] (F) -- (G);
    \end{tikzpicture}
    \phantom{and}
    \begin{tikzpicture}
      \node (E) at (0, 0) {$k$};
      \node (F) at (1.5, 1) {$K$};
      \node (G) at (0.5, 2) {$L$};

      \draw[arrow] (E) -- (F);
      \draw[dashed-arrow] (E) -- (G);
      \draw[arrow] (F) -- (G);

    \end{tikzpicture}
    \phantom{and}
    \begin{tikzpicture}
      \node (E) at (0, 0) {$k$};
      \node (F) at (1.5, 1) {$K$};
      \node (G) at (0.5, 2) {$L$};

      \draw[dashed-arrow] (E) -- (F);
      \draw[arrow] (E) -- (G);
      \draw[arrow] (F) -- (G);

    \end{tikzpicture}
    \caption{The different configurations with triangles.}
    \label{fig:triangles}
  \end{figure}

\paragraph{Description of the framework.} The general strategy of Bosma, Canon,
and Steel to compute and maintain the data of several finite fields and
compatible embeddings between them is to ensure that some properties are
conserved at all times. In all the section, we fixe $p\in\mathbb{N}$ a prime
number and $q$ a power of $p$. The results are true for an arbitrary power of
$p$, although the applications and the implementations we have in mind are
relevant for $q=p$. If $k=\mathbb{F}_{q^m}$ is a finite extension of
$\mathbb{F}_q$, we define
\[
  \partial (k) = \left[ k:\mathbb{F}_q \right] = m
\]
the extension degree over $\mathbb{F}_q$.
\begin{defi}[Lattice of compatibly embedded finite fields]
  \label{defi:lattice-bcs}
  Let $q$ be a power of $p$ a prime number and $\mathbb{F}_q$ the finite field
  with $q$ elements. The couple
\[
  \Lambda = (\mathcal K, \Phi),
\]
where $\mathcal K$ is a collection of finite fields sharing the same base field
$\mathbb{F}_q$ and $\Phi$ is a collection of embeddings between members of
$\mathcal K$,
is called a \emph{lattice of compatibly embedded finite fields} if the following
conditions are satisfied:
\begin{description}
  \item[CE1 (unicity)] for each pair $(k, K)$ of elements in $\mathcal K$, there exists
    at most one corresponding embedding $\embed{k}{K}\in\Phi$.
    \item[CE2 (reflexivity)] For each $k\in \mathcal K$, the identity map
    $\Id_k=\embed{k}{k}$ is in $\Phi$.
  \item[CE3 (base subfield)] There is exactly one $k\in \mathcal K$ such that
    $\partial(k)=1$, and for all $K\in \mathcal K$, there exists $\embed{\mathbb{F}_q}{K}\in\Phi$.
  \item[CE4 (invertibility)] If $k\emb K$ and $\partial(k)=\partial(K)$, then $K\emb k$ and
    $\embed{K}{k}=\embed{k}{K}^{-1}$.
  \item[CE5 (transitivity)] For any triple $(k, K, L)$ of elements in $\mathcal
    K$, if $E\emb F\emb G$ then $E\emb G$ and
    $\embed{E}{G}=\embed{F}{G}\circ\embed{E}{F}$.
  \item[CE6 (intersections)] For each $k, K, L\in \mathcal K$ such that $K\emb L$ and
    $k\emb L$, there exists $I\in \mathcal K$ such that
    $\partial(I)=\gcd(\partial(k), \partial(K))$
    and $I\emb k$, $I\emb K$.
\end{description}
\end{defi}
Conditions CE1 to CE5 are quite natural, some of them are technical and
are not important in our implementation. For example, Condition~CE3
is totally free in our implementation because we work with $q=p$ and the
elements of our fields are represented by polynomials over
$\mathbb{F}_p=\mathbb{Z}/p\mathbb{Z}$, thus the subfield
$\mathbb{F}_p\subset\mathbb{F}_{p^m}$ is always represented by the constant
polynomials and the embedding is trivial. Nevertheless, Condition~CE6 is
important both on a theoretical and on an implementation point of view: it
ensures that the implicit embeddings between common subfields are made explicit,
in order to prevent any possible future incompatibility, and it forces the
computer algebra software to compute extra embeddings that the user might not
need. Assume that we have embedded $\mathbb{F}_{p^4}$ and $\mathbb{F}_{p^6}$
into $\mathbb{F}_{p^{12}}$, there is now an implicit isomorphism between the two
copies of the quadratic subfield $\mathbb{F}_{p^2}$ in $\mathbb{F}_{p^4}$ and
$\mathbb{F}_{p^6}$.
\begin{center}
      \begin{tikzpicture}
        \node (4) at (0, 0) {$\mathbb{F}_{p^{4}}$};
        \node (6) at (2, 0) {$\mathbb{F}_{p^{6}}$};
        \node (12) at (1, 2) {$\mathbb{F}_{p^{12}}$};

      \draw[arrow] (4) -- (12);
      \draw[arrow] (6) -- (12);
%      \draw[dashed-arrow] (F) -- (G);
    \end{tikzpicture}
\end{center}

Under the conditions described in Definition~\ref{defi:lattice-bcs}, we
can add new embeddings in our lattice $\Lambda$ without compromising the
compatibility of the lattice.

\subsection{Implementation in Nemo}
%%
