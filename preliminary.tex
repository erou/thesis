Throughout all this document, we will use a lot of results from algebra. This
chapter is here to sum up these results and try to maintain the illusion that
this thesis is self-contained. The reader familiar with the notions of finite
fields, algebraric function fields, complexity model may very well skip this
chapter.

\section{Finite fields}
Finite fields are ubiquitous in cryptography and coding theory, probably because
their field structure, a rigid one, allows to understand how they work, and
their finiteness makes them easier to represent on a computer. They are also
everywhere in this thesis, and are probably on almost every paper I
wrote on during these last three years. They are quite important.

\section{Algebraic function fields}
\section{Complexity models}
In algorithmic, it is of central interest to understand how our algorithms
\emph{scale}, \ie to understand how they perform if the size of the input is
getting larger and larger. Complexity theory studies this phenomenon and give us models
of computation in order to quantify the behaviour of our algorithms. Depending
on the situation, not all models are relevant, and one has to balance between
the concreteness of a model and its ease to use. An extreme viewpoint is to
specify an operating system with a compiled programming language and to compare
the running time or the memory requirements between algorithms. The advantage of
such a model is that it is very concrete, but it is also its main disadvantage
because it makes the model hard to use. Thus, there exist other models of
\emph{idealized} computer, such as Turing machines~\cite{Papadimitriou03},
random access machines, or the algebraic complexity model.
We use the latter, that we present in more details in the next section.

\subsection{Algebraic complexity}

This model is widely presented in~\cite{BCS13}, we only give a brief
presentation of the subject. This model assumes that we use an abstract computer
that is able to perform operation in some base field $\K$ at a constant, unit
cost. We also assume that accessing the memory of the computer is free.
Algebraic complexity is especially useful with algorithms dealing with
algebraic structures. This is very handy for us, since
we usually work with algebras
\[
  (\A, +, \times, \cdot)
\]
over some base field $\K$. As an example, with this model, the complexity of an
addition in the extension field
\[
  \mathbb{F}_4 = \mathbb{F}_2[T]/(T^2+T+1) = \mathbb{F}_2(x)
\]
where $x=\bar T$, if elements are represented in the basis $\left\{ 1, x
\right\}$, is $2$, because we only need $2$ additions in $\mathbb{F}_2$ to
perform an addition in $\mathbb{F}_4$. Indeed, if
\[
  a = a_0 + a_1x\in\mathbb{F}_4
\]
and
\[
  b = b_0 + b_1x\in\mathbb{F}_4,
\]
we have
\[
  a+b = (a_0+b_0)+(a_1+b_1)x.
\]
In the case of a multiplication, we have
\[
  ab = (a_0b_0+a_1b_1) + (a_0b_1+a_1b_0)x,
\]
so the complexity of a multiplication in $\mathbb{F}_4$ is (at least with this
formula) $6$, because we need
$4$ multiplications and $2$ additions in $\mathbb{F}_2$.
Ah ah AH ah of of et là

