In Chapter~\ref{chap:bilinear}, we have seen the notions of bilinear complexity
and symmetric bilinear complexity. We now investigate even stronger
notions of symmetry, allowing to have very short representations of a bilinear
map.

\minitoc

% TODO
% ====
%
% Find a nice picture to put here to illustrate something in link with the
% chapter.

\clearpage
\section{Symmetric and hypersymmetric fomulas}
% Table of content
% ================
%
% - Recall of the definition of symmetric
% - Existence and lemma for the symmetric case
% - non degenerate bilinear form, link with the trace, but not only
% - Link between symmetric and hypersymmetric in smaller dimension
% - Galois invariance
% - Comment for the case of the particular algebras we study, what is known and
%   what is not
%
% Comment
% =======
%
% Comment about the trisymmetric formulas, it is true that F_4/F_2 can be
% represented by a trisymmetric formula but it is not the case for F_8/F_2. We
% should check the lemma saying something on the existence of the trisymmetric
% decomposition, but it is probably just simpler.

Let $\K$ be a finite field, $V_1$, $V_2$ and $W$ three finite-dimensional $\K$-vector
space and
\[
  \Phi:V_1\times V_2\to W
\]
a bilinear map. Recall Definition~\ref{defi:bilinear-formula}:
\[
  \Phi(x, y) = \sum_{j=1}^t\varphi_j(x)\psi_j(y)w_j,
\]
where for all $1\leq j\leq t$, $\varphi_j\in V_1^\vee$ and $\phi_j\in V_2^\vee$ are linear forms and
$w_j\in W$ is a vector, is called a \emph{bilinear formula} of length $t$. If
the spaces $V_1$ and $V_2$ are equal and if the bilinear map $\Phi$ is
symmetric, \ie if for all $x, y\in V$
\[
  \Phi(x, y) = \Phi(y, x),
\]
we can investigate the existence of formulas satisfying the same condition of
symmetry, \ie formulas where for all $1\leq j\leq t$, $\varphi_j=\psi_j$,
resulting in \emph{symmetric} bilinear form:
\[
  \Phi(x, y) = \sum_{j=1}^t\varphi_j(x)\varphi_j(y)w_j.
\]
In fact, we can define other interesting types of symmetries, but it is useful
to first generalize the notions that we saw in Chapter~\ref{chap:bilinear} to
higher dimensions.
\begin{defi}[Multilinear formula]
Let $V_1, V_2, \dots, V_s$ and $W$ be $s+1$ finite-dimensional $\K$-vector
spaces and
\[
  \Phi:V_1\times V_2\times\dots\times V_s\to W
\]
an $s$-multilinear map. A \emph{multilinear formula}, or \emph{multilinear
decomposition}, or \emph{multilinear algorithm} of length $t$ for $\Phi$ is a
collection of $s\times t$ linear forms $\varphi_1^{(1)}, \varphi_2^{(1)}, \dots,
\varphi_t^{(1)}\in V_1^\vee$ up to $\varphi_1^{(s)}, \varphi_2^{(s)}, \dots,
\varphi_t^{(s)}\in V_s^{\vee}$ and $t$ vectors $w_1, \dots, w_t$, such that for all $x_1\in V_1, \dots, x_s\in
V_s$, we have
\[
  \Phi(x_1, \dots, x_s) =
  \sum_{j=1}^t\varphi_j^{(1)}(x_1)\dots\varphi_j^{(s)}(x_s)w_j.
\]
\end{defi}
\begin{defi}[Multilinear complexity]
Let $V_1, V_2, \dots, V_s$ and $W$ be $s+1$ finite-dimensional $\K$-vector
spaces and
\[
  \Phi:V_1\times V_2\times\dots\times V_s\to W
\]
an $s$-multilinear map. The \emph{multilinear complexity} $\mu(\Phi)$ of $\Phi$ is the
minimal length $t$ of a multilinear formula for $\Phi$.
\end{defi}
As in the case of bilinear complexity, the multilinear complexity $\mu(\Phi)$ of a
multilinear map $\Phi$ can also be defined as the rank of the tensor in 
\[
  V_1^\vee\otimes\dots\otimes V_s^\vee\otimes W
\]
corresponding to $\Phi$.
% Comment
% =======
%
% There should be an example of that for the bilinear case, such that now it is
% not so obscure.

\section{Algorithmic seach in small dimension}
\section{Asymptotic complexities}
