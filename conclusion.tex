In this thesis, we invastigated the arithmetic of finite field extension from
two different angles, the arithmetic of a single finite field extension, and the
arithmetic of a lattice of compatibly embedded finite fields.

We studied an alternative, more rigid, kind of bilinear complexity, called
hypersymmetric bilinear complexity. Our algorithm to find trisymmetric formulas
is very useful to understand and exhibit formulas, but its
complexity is prohibitive. We were able to reciprocate the last known records in
the dimension $k$ in which formulas were found, but we were not able to go
beyond that point. With more optimization, one could hope to push a
little further the computations, but it is likely that new methods have to be
found to really make a breakthrough. Asymptocically, we proved that the
hypersymmetric complexity of the multiplication in
\[
  \mathbb{F}_{p^{k}}
\]
is linear in the degree $k$ of the extension, just like with bilinear
complexity. However, our approach is clearly not optimal, since we obtain the
result as a corollary from the same result in higher dimension. One could hope
to obtain better bounds on the hypersymmetric complexity by using an
\emph{ad hoc} proof.

We implemented a the Bosma-Canon-Steel framework in Nemo, and we introduced a
new idea to produce a lattice of
compatibly embedded finite fields, based on both Conway polynomials and the
Bosma-Canon-Steel framework. Conway polynomials are used in many computer
algebra systems, and the Bosma-Canon-Steel framework is used in Magma (and Nemo).
Our idea exploits new techniques and is thus
interesting in itself. Furthermore, it also leads to a new family of (standard)
polynomials that can be used to define finite fields. Nevertheless, this new
family has no practical impact at the moment. Indeed, we can prove that
computing these polynomials is essentially equivalent to the computation of
Conway polynomials. Indeed, with our polynomials, one can recover a standard
solution $\alpha_m$ of~\eqref{eq:h90-kummer} and deduce the value
\[
  (\zeta_{p^a-1})^a.
\]
Then, by taking a $a$-th root, which is done in polynomial time in $m$, one can
find $\zeta_{p^a-1}$ and recover the associated Conway polynomial by computing a
minimal polynomial. This means that an efficient algorithm to compute our
standard polynomials would lead to an efficient algorithm to compute Conway
polynomials, which would be unexpected. We thus do not have great hope to find
such an algorithm, however the implemention presented in
Section~\ref{sec:implementation-std-lattices} is not the only possible way to
exploit our definitions. Indeed, it could be possible to loosen the assumption
that a cyclotomic lattice exists, and thus find a middle ground between the
rigidity of Conway polynomials and the flexibility of the Bosma-Canon-Steel
framework, for example by lazily computing the roots of unity only when needed.
An orthogonal line of work would be to construct a complete lattice of
compatibly embedded finite fields, \ie working even with degrees that are not
coprime with the characteristic $p$ of the base field $\K$.
